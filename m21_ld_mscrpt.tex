\documentclass[review]{elsarticle}
\usepackage{lineno,hyperref}
%\usepackage{apacite}
\modulolinenumbers[5]
%\usepackage{natbib}

\journal{Neuropsychologia}

%%%%%%%%%%%%%%%%%%%%%%%
%% Elsevier bibliography styles
%%%%%%%%%%%%%%%%%%%%%%%
%% To change the style, put a % in front of the second line of the current style and
%% remove the % from the second line of the style you would like to use.
%%%%%%%%%%%%%%%%%%%%%%%

%% Numbered
%\bibliographystyle{model1-num-names}

%% Numbered without titles
%\bibliographystyle{model1a-num-names}

%% Harvard
%\bibliographystyle{model2-names.bst}\biboptions{authoryear}

%% Vancouver numbered
%\usepackage{numcompress}\bibliographystyle{model3-num-names}

%% Vancouver name/year
%\usepackage{numcompress}\bibliographystyle{model4-names}\biboptions{authoryear}

%% APA style
 \bibliographystyle{model5-names}\biboptions{authoryear}

%% AMA style
%\usepackage{numcompress}\bibliographystyle{model6-num-names}

%% `Elsevier LaTeX' style
% \bibliographystyle{elsarticle-num}
%%%%%%%%%%%%%%%%%%%%%%%

\begin{document}
	
\begin{frontmatter}

%\title{Elsevier \LaTeX\ template\tnoteref{mytitlenote}}
%\tnotetext[mytitlenote]{Fully documented templates are available in the elsarticle package on \href{http://www.ctan.org/tex-archive/macros/latex/contrib/elsarticle}{CTAN}.}

\title{Differential patterns of brain activity are correlated with sensitivity to morphological structure}

%%% Group authors per affiliation:
%\author{Elsevier\fnref{myfootnote}}
%\address{Radarweg 29, Amsterdam}
%\fntext[myfootnote]{Since 1880.}

%% or include affiliations in footnotes:
\author[mymainaddress]{Joanna Morris\corref{mycorrespondingauthor}}
\cortext[mycorrespondingauthor]{Corresponding author}
\ead{jmorris6@providence.edu}

\author[mymainaddress]{Emma Kealey}
\ead{ekealey@providence.edu}

\address[mymainaddress]{Department of Psychology, Providence College, Providence, RI, USA}


\begin{abstract}
Skilled reading requires us to rapidly retrieve a word's meaning from memory based on a brief exposure to a sequence of arbitrary, abstract visual symbols. Within the lexicon, the mapping from form to meaning takes place at the level of morphology, and rapid semantic retrieval is facilitated by lexical representations that are precise, specific, redundant and flexible. Given that both lexical quality and sensitivity to morphological structure are key components of reading success, can we identify patterns of brain activity that reflect individual differences in these characteristics? To answer this question, we collected event-related potential and response time data from participants as they completed a lexical decision task featuring complex words that varied in the frequency of their morphological components. These data allowed us to determine individual differences in sensitivity to morphological structure. We also quantified individual differences in lexical quality (LQ) via estimates of vocabulary size and spelling performance, both of which are correlated with the precision and stability of lexical representations. We found differences between high LQ and low LQ participants in the N250 ERP component which has been hypothesised to reflect the processing sub-lexical orthographic units. These findings suggest that individual differences in lexical quality and sensitivity to morphological structure are reflected in distinct patterns of brain activity, and opens the avenue for further work to investigate the perceptual and cognitive processes that underlie these differences.
\end{abstract}


%%Each keyword shall be separated by a \sep command. msc classifications shall be provided in the keyword  environment with the commands \MSC. \MSC accepts an optional argument to accommodate future revisions. eg., \MSC[2008]. The default is 2000.

\begin{keyword}
 morphology\sep event-related potentials\sep visual statistical learning \sep individual differences
\end{keyword}

\end{frontmatter}

\linenumbers

\section{Introduction}

Reading can be defined as `the process of gaining meaning from print' \citep[][p. 34]{raynerHowPsychologicalScience2001}. Words are the primary means by which language conveys meaning, hence learning to decode printed words is an important part of learning to read \citep[e.g.,][]{castlesEndingReadingWars2018}.  Although all writing systems consist of abstract visual symbols that are mapped onto spoken language units, orthographies differ with respect to \textit{which} spoken unit is mapped to a visual symbol; alphabetic systems map symbols to phonemes (the individual sounds of the language), syllabaries  map them to syllables, and logographies to words or morphemes \footnote{Chinese is typically cited as the canonical example of a logographic system, but although it may have originated as a logography, in its current incarnation it is more accurately classified as a morpho-syllabic system \citep{raynerHowPsychologicalScience2001}}.  The choice of spoken linguistic unit has consequences for how reading proceeds. %do we assume readers know what phonemes are? 

Because alphabets map visual symbols to individual speech sounds rather than to units of meaning,  learning the relationship between letters and sounds is necessary and nonnegotiable for beginning readers of any alphabetic writing system. However, learning this mapping can be difficult because languages that are written using alphabets differ in terms of orthographic depth, or the consistency of the mapping between letters and sounds. In a strictly alphabetic or ‘shallow’ orthography, one grapheme (letter or symbol) corresponds to one phoneme (sound), but in a ‘deep’ orthography the mappings between graphemes and phonemes  are many-to-many rather than one-to-one.

English writing uses a ‘deep’  orthography where the grapheme-to-phoneme mappings are  context-sensitive, and  can be affected by factors such as letter position and and the identity of neighbouring graphemes. For example, at the syllabic level, post-vocalic consonants influence vowel pronunciation---the $\langle a\rangle$ in `hat’ is pronounced differently from the  $\langle a\rangle$ in ‘hate’ as a result of the addition of a syllable final ‘e’---an example of a complex non-adjacent dependency. This leads to a situation in which English spelling is irregular at the level of individual graphemes and phonemes, but much more regular at the level of the spoken and orthographic rimes\footnote{A rime is the part of a syllable which consists of its vowel and any consonant sounds that come after it.} \citep{treimanSpecialRoleRimes1995}. Because both graphemes and orthographic rimes play an important role in the pronunciation of printed words, beginning readers must learn how to recognize both types of written units and associate them with their spoken counterparts. % I think we need to expand more on Rimes here and their relevance. As illustrated above, the grapheme 'a' can be mapped to multiple vowel sounds (phonemes) depending on the surrounding graphemes. While 'a' itself is not consistent, if you look at the Rime of the word the mappings are more consistent. The 'a' in words with similar rimes such as late, mate, hydrate, and operate are all mapped to the same phoneme. 

%In deep orthographies, beginning readers must learn both the grapheme <>phoneme mappings AND must learn the patterns that dictate the mappings. In English, this would be the orthographic rimes.

In addition to having a deep orthography, the English writing system also exhibits a trade-off between phonological explicitness and morphological transparency. The trade-off can be illustrated by the observation that the meanings of groups of letters corresponding to stems are usually preserved in their derivations (e.g., ‘\textit{heal}’ \textrightarrow ‘\textit{health}’). This preservation of the stem spelling makes morphological relatedness more transparent in written language, thus facilitating written word recognition. In addition to keeping stems consistent across derivations, English also exhibits highly faithful affixes in that groups of letters corresponding to affixes alter the meanings of stems in consistent ways (e.g., \textit{‘help‑less}’, ‘\textit{home‑less}’, ‘\textit{worth‑less}’ \textrightarrow ‘‑\textit{less}’ = ‘\textit{without}’). This consistency in both the spelling and meaning of affixes contributes to the grapheme >< morpheme mappings by creating reliable and salient patterns for readers to use during word recognition. This leads to consistency in the grapheme-to-morpheme mapping at the expense of inconsistency in grapheme-to-phoneme correspondence \citep{zieglerReadingAcquisitionDevelopmental2005,rastlePlaceMorphologyLearning2019}. % what do the -less words show here? I see the point that affixes are consistent in English (they are spelled the same and have generally the same meaning across words). However, how does that contribute to the point that this consistency comes **at the price** of grapheme<>phoneme consistency?
%other heal / health examples would be autonomy vs automatic, part / partial, 

% the -less words illustrate that english is faithful / consistent in (at least some of it's) . -less has consistent orthographic form AND consistent meaning. It reverses whatever word it is attached to without question. This consistency with affixes should, in theory, make them more salient / easier to extract. As they occur more consistently and in the same contexts/outcomes, the pattern (statistical probablitiy becomes more reliable and thus more accessible to learners).

Because of the complexity of English orthography, English-speaking children must acquire a rapid and flexible word recognition system that can accommodate mappings from symbols to phonemes, to syllables and sub-syllabic units such as onsets and rimes, as well as to units of meaning such as morphemes. The context-sensitive, multi-representational character of the English writing system means that  phonological awareness, syllabic awareness, and sensitivity to morphological structure impact reading ability as children progress through school \citep{mahonyReadingAbilitySensitivity2000}.  

Children are rarely explicitly taught all of the relationships between graphemes, phonemes, syllables and morphemes that they must know in order to become skilled readers; some of this knowledge must be acquired implicitly via the process of  statistical learning (SL), the mechanism by which we extract repeated patterns from our environment. SL enables the detection of probabilistic correspondences between letters and phonemes as well as the detection of regularities that exist among letters. For example, an orthographic unit, such as a word (e.g. `bird') or affix (e.g. `-er'), is defined by higher transitional probabilities between letters within a unit than between letters spanning two distinct units. Sensitivity to probabilistic orthography-to-phonology mappings follows a developmental trajectory throughout the elementary and middle school years \citep{treimanSpellingStatisticalLearning2006} and  it is  a significant predictor of reading ability. However, there is substantial variation in SL performance among both children and adults \citep{arciuliStatisticalLearningRelated2012}.   Do these differences in SL contribute to individual differences in  reading skill?

\subsection{Lexical quality}
%I think we're missing some connections here. My understanding is that our argument is 

%1. Children (beginning readers) must learn a lot of mappings to become skilled readers of English because english mappings are highly context dependent. Children are not explicitly taught every single context rule. Some of it must be inferred via SL. Thus, is it the case that differences in SL contribute to differences in reading skills?

%2. Once they are skilled readers, children access three representations of a simultaneously: sound, meaning, written form. Being able to access these three representations simultaneously, automatically, and rapidly is the result of having developed an expert mental lexicon. We define 'expert' as having high quality lexical representations.

%2.1 So what do we mean when we say "High Quality Lexical Representations"? 
%Quality refers to XXX [this section needs a little help]

%2.2 Skilled readers show variability in measures of written proficiency and word recognition. These differences may be related to differences in lexical quality.

%2.2.1 High quality representations allow for rapid identification of a word AND minimize activation of competing candidates. This is done without needing to recruit attentional or computational resources. BC these resources aren't involved in word recognition, they can be used for higher order comprehension of connected discourse.

%2.3 We can measure lexical quality with tests of vocabulary size and spelling tests.

Reading for comprehension involves more than merely recognizing words. It depends on knowledge of grammar; we know that ``the man bit the dog" is newsworthy while ``the dog bit the man" is less so.  It also requires an understanding of social norms such that we interpret a the question ``can you pass the salt?" as a request to do so, not a query about our physical capabilities.  That said, skill in reading comprehension rests to a considerable extent on knowledge of words.

For literate speakers of a language, knowing a word means being familiar with three distinct word representations— a word's sound, its meaning and its written form. For skilled readers,  all three representations are activated so quickly and seamlessly that they are perceived as a single entity. The rapid, automatic, and synchronous activation of a multi-component lexical representation is afforded by skilled readers having developed an `expert' mental lexicon characterized by high quality lexical representations.

\citet{perfettiReadingAbilityLexical2007} defines `lexical quality'  as `the extent to which the reader's knowledge of a given word represents the word's form and meaning constituents and knowledge of word use that combines meaning with pragmatic features" (p. 359). That is, lexical quality refers to a reader's knowledge of a word's form (grammatical class, spelling, pronunciation), meanings, and how the word is influenced by pragmatics
High-quality lexical representations, then, are those that are precise and flexible \citep{perfettiReadingAbilityLexical2007}.  Precision allows readers to efficiently identify a word and minimize competition from words with overlapping morphemes or spellings. Flexibility allows for new interpretations and for the system to recover from mistakes or variations in spelling, pronunciation, and use. High quality representations have fully specified orthography, phonology that is redundant insofar as it is both stored as part of the word  as well as recoverable from orthography-to-phonology mappings, meaning representations that are tuned to context, a complete morpho-syntactic specification and a strong bond among each of these constituent representations \citep{perfettiLexicalQualityHypothesis2002, perfettiReadingAbilityLexical2007, perfettiLexicalQualityRevisited2017} %Precise representations (as compared to vague representations) facilitate recognition by minimizing activation of non-target lexical representations. Flexibility allows for new interpretations and for the system to recover from mistakes or variations in spelling, pronunciation, and use. 
  
Even skilled readers show great variability in measures of written language proficiency as well as in experimental measures of word recognition such as response times are error rates in a lexical decision task \citep{andrewsIndividualDifferencesSkilled2012}, and these proficiency differences  may be related to differences in lexical quality.  Andrews and colleagues \citep{andrewsLexicalExpertiseReading2009, andrewsLexicalPrecisionSkilled2010, andrewsIndividualDifferencesSkilled2012, andrewsMorphologicalPrimingStronger2013}  argue that skilled readers rely on precise lexical representations that allow for the rapid identification of the intended word with minimal activation of competing candidates, and without the need to recruit limited attentional and computational resources.  These resources can then be directed to the higher-order inferential processes required for the comprehension of connected discourse.

We can measure lexical precision using the masked form priming paradigm, in which  responses to word targets are influenced by primes that share formal similarity with the target, (e.g. `\textit{clam}' primes `\textit{calm}').  The finding that this priming effect is limited to words, but does not appear for non-word targets is taken as evidence that masked form priming reflects pre-activation of the lexical representation of the target word. \citep{andrewsLexicalPrecisionSkilled2010,forsterRepetitionPrimingFrequency1984}.   How do individual differences in the quality of lexical representations affect the process of reading?  When orthographic representations are precise ( `\textit{trail}' $\neq$ `\textit{trial}'), they are less vulnerable to being activated by a prime containing a mismatching letter \citep{forsterBodiesAntibodiesNeighborhooddensity1994}.  

\citet{yapIndividualDifferencesJoint2009} propose that vocabulary knowledge (i.e., knowledge of word forms and word meanings) is a reasonable proxy for lexical quality as there is evidence that vocabulary size is correlated with the precision and stability of lexical representations \citep{perfettiReadingAbilityLexical2007, perfettiLexicalQualityHypothesis2002, kiddIndividualDifferencesStatistical2016,kinoshitaAdditiveInteractiveEffects2006, kinoshitaHowLexicalDecision2006}  Along with vocabulary size, spelling performance provides a good index of lexical quality because while many reading tasks can be successfully achieved using only partial information, accurate spelling requires precise, word-specific knowledge \citep{perfettiRepresentationProblemReading1992}.

\subsection{Individual differences in complex word recognition}

Psychologists who study reading work within an experimental tradition in which they develop models based on the average responses of a sample of individuals to a manipulation of environmental variables \citep{cronbachTwoDisciplinesScientific1957}.This approach rests on the implicit assumption that the cognitive architecture of the reading system is uniform across skilled readers. But if readers differ in the quality of their lexical representations, and if these differences lead to differences in the perceptual and cognitive processes underlying word reading, this assumption may not hold.

{\renewcommand\&{and}\citet{andrewsMorphologicalPrimingStronger2013}} argue that although this assumption has been useful  in  establishing the general principles of word reading, further progress requires us to consider the role of individual differences.  This approach has already been shown to bear fruit in terms of providing clarity on a  phenomenon that has generated  considerable discussion, morpho-orthographic priming.  Studies of morphological priming have consistently shown that morphologically complex words facilitate lexical decision responses to their stem morphemes across a range of languages, suggesting that morpheme representations shared by primes and targets are activated early in the process of word recognition.  In the masked priming paradigm, a number of studies have found that priming is not limited to primes that overlap in meaning (e.g. `\textit{payment}'-`\textit{pay}' ), but also extend to pseudo-complex words (e.g. `\textit{corner}'-`\textit{corn}').  This evidence has been taken as support for `morpho-orthographic decomposition’ accounts of morphological processing \citep{rastleBrothMyBrother2004, rastleMorphologicalDecompositionBased2008} which assume an early orthographically-driven segmentation process that parses complex words into affixes and stems.

However, these findings are at odds with a  meta-analysis of 16 studies comparing masked priming for transparent and opaque words showed significantly stronger priming effect for transparent than opaque words \citep{feldmanEarlyMorphologicalProcessing2009}. Moreover, equivalence in the degree of priming between semantically transparent morphological relatives versus those that are share a semantically opaque or even pseudo-morphological relationship poses a problem for lexical models that rely on statistical learning \citep{plautAreNonsemanticMorphological2000, baayenAmorphousModelMorphological2011}. These models capture sensitivity to morphological structure through learned associations between orthography and meaning; morphological relationships emerge as a consequence of these associations, and  produce graded effects of morphological structure such that transparent pairs, which share both orthographic and semantic features yield stronger priming than opaque pairs that share only orthography.    

By examining individual differences between readers, {\renewcommand\&{and}\citet{andrewsMorphologicalPrimingStronger2013}}  were able to shed light on these discrepancies in the literature.  If morphology represents a learned sensitivity to the systematic relationships among the surface forms of words and their meanings, differences between English readers in their sensitivity to orthographic units and to the semantic structure of words could  yield differences in the extent to which their lexicons become morphologically structured.  {\renewcommand\&{and}\citet{andrewsMorphologicalPrimingStronger2013}} found that individual differences in in orthographic and semantic processing (indexed by scores on spelling and vocabulary tests respectively )were associated with systematic differences in participants' responses to morphological priming.  Individuals with higher vocabulary than spelling scores (`semantic' profile) showed strong priming for transparent pairs but minimal priming for opaque pairs. In contrast, individuals with higher spelling than vocabulary scores (‘orthographic profile’ indicating precise orthographic representations of words), showed equivalent priming for opaque and transparent pairs \citep{andrewsMorphologicalPrimingStronger2013}.

{\renewcommand\&{and}\citet{andrewsMorphologicalPrimingStronger2013}} argue that a `semantic' reading profile may reflect a holistic and contextually-driven reading strategy with greater sensitivity to conceptual and semantic representations than to orthographic patterns.  In contrast, an `orthographic' profile may be associated with a bottom-up reading strategy focused on extracting  frequently occurring orthographic patterns such as common affixes, regardless of their morphological status—morpho-orthographic decomposition.  . 
 
A focus on morpho-orthographic units may be necessary to enable the developing reader to discover the morphological regularities that characterise the mapping between orthography and meaning.  But how does the developing reader learn what letter sequences form morphological units in written text?  {\renewcommand\&{and}\citet{rastleMorphologicalDecompositionBased2008}} propose that this is accomplished via sensitivity to the statistical structure of letter sequences, i.e. marking low probability sequences as containing boundaries and  grouping high- probability sequences into single units. If statistical learning drives morphological segmentation, individual differences in sensitivity to statistical patterns may lead to differences in the strength and precision of morphemic representations,  sensitivity to the morphological structure of words,  and the use of such structures in word recognition. 

Although it has been shown that individual differences in the capacity for statistical learning are linked with variability in word reading accuracy \citep{arciuliReadingStatisticalLearning2018, arciuliStatisticalLearningRelated2012} we still do not have a clear understanding of the mechanisms that give rise to these associations. What role do domain-general mechanisms of implicit statistical learning play in extracting morphological structure and establishing sub-lexical morpho-orthogrphic to lexico-semantic mappings?  To what extent do individual differences in implicit statistical learning impact this process? The first steps in answering these questions are to determine the extent to which differences in readers’ sensitivity to the internal structure of words is reflected in differences in the patterns of brain responses to words that differ in morphological structure. 

\subsection{The current study}

The goal of the study was to examine the extent to which  individual differences in sensitivity to the internal structure of words is reflected in differences in  patterns of brain responses. To examine potential differences in sensitivity to morphological structure, we manipulated the cumulative root frequency, morphological family size and morphological complexity of non-word targets, and measured the amplitude of the N250 component as participants completed a lexical decision task.  

Complex word recognition is influenced by many different kinds of lexical and sub-lexical frequency  information, and morpheme frequency effects have often been used to test whether derived words activate morphemic representations; three type of frequency information have been shown to be particularly important—surface frequency, cumulative root or cumulative root frequency and family size. Surface frequency refers to the frequency of the whole-word string, family size refers to the type count of morphologically related words, and cumulative root frequency, is the cumulative token frequency of all morphologically related family members. 

The cumulative root frequency effect refers to the finding that words with high cumulative root frequency are responded to faster and more accurately than words with low cumulative root frequency when surface frequency is controlled \citep{taftRecognitionAffixedWords1979}  Taft (1979).  The family size effects refers to the finding that  lexical decision response times to words with larger family sizes (i.e., appearing as a constituent in larger numbers of derived words and compounds) are faster than for words with smaller family sizes. The cumulative root frequency effect only occurs in words with productive affixes, whereas the family size effect occurs regardless of affix productivity. Semantically transparent family members drove family size effects and the family size effect was largely absent or attenuated in semantically opaque family members. Both family size and  cumulative root frequency reflect sensitivity to mophological structure, but  there is not consensus in the field regarding their relationship. 
The family size effect has hypothesized to reflect the activation of semantic representations as it sensitive to semantic transparency.  In contrast, the cumulative root frequency effect is affected by affix productivity, hence it has been hypothesized to the activation of morphological form representations.

In this study we chose to examine  non-word processing in order to manipulate the orthographic properties of stimuli with greater precision than is possible using real words.  We examined the effect of morphological structure on non-word recognition via the morpheme interference effect, or non-word complexity effect.  This effect occurs when non-words that are created by combining existing morphemes (e.g., GASFUL) are rejected more slowly in a lexical decision task than words that do not have lexical structure (e.g., GASFIL).  We hypothesized that the amplitude of the N250 component to  these non-words would be modulated by participant reading style, as well as by morphological family size, and/or cumulative root frequency, and specifically that  this modulation of the n250 response by morphological structure would be greater for readers with greater sensitivity to the morphological structure of words, or better `lexical quality'.  Out dependent measure was the magnitude N250 response.  The N250 is a negative-going wave that starts as early as 110 milliseconds and peaks around 250 milliseconds. {\renewcommand\&{and}\citet{holcombTimeCourseVisual2006}}  suggest that the N250 reflects the process that maps sub-lexical orthographic and phonological representations onto whole word representations.

\section{Methods}


\section{Results}



\section{Discussion}
	
% \section*{References}


\bibliography{PCHPL}

\end{document}
