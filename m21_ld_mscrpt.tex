\documentclass[review]{elsarticle}

\usepackage{lineno,hyperref}
\modulolinenumbers[5]
%\usepackage{natbib}

\journal{Journal of \LaTeX\ Templates}

%%%%%%%%%%%%%%%%%%%%%%%
%% Elsevier bibliography styles
%%%%%%%%%%%%%%%%%%%%%%%
%% To change the style, put a % in front of the second line of the current style and
%% remove the % from the second line of the style you would like to use.
%%%%%%%%%%%%%%%%%%%%%%%

%% Numbered
%\bibliographystyle{model1-num-names}

%% Numbered without titles
%\bibliographystyle{model1a-num-names}

%% Harvard
%\bibliographystyle{model2-names.bst}\biboptions{authoryear}

%% Vancouver numbered
%\usepackage{numcompress}\bibliographystyle{model3-num-names}

%% Vancouver name/year
%\usepackage{numcompress}\bibliographystyle{model4-names}\biboptions{authoryear}

%% APA style
 \bibliographystyle{model5-names}\biboptions{authoryear}

%% AMA style
%\usepackage{numcompress}\bibliographystyle{model6-num-names}

%% `Elsevier LaTeX' style
% \bibliographystyle{elsarticle-num}
%%%%%%%%%%%%%%%%%%%%%%%

\begin{document}

\begin{frontmatter}

\title{Elsevier \LaTeX\ template\tnoteref{mytitlenote}}
\tnotetext[mytitlenote]{Fully documented templates are available in the elsarticle package on \href{http://www.ctan.org/tex-archive/macros/latex/contrib/elsarticle}{CTAN}.}

%%% Group authors per affiliation:
%\author{Elsevier\fnref{myfootnote}}
%\address{Radarweg 29, Amsterdam}
%\fntext[myfootnote]{Since 1880.}

%% or include affiliations in footnotes:
\author[mymainaddress]{Joanna Morris\corref{mycorrespondingauthor}}
\cortext[mycorrespondingauthor]{Corresponding author}
\ead{jmorris6@providence.edu}

\author[mymainaddress]{Emma Kealey}
\ead{ekealey@providence.edu}

\address[mymainaddress]{Department of Psychology, Providence College, Providence, RI, USA}


\begin{abstract}
Skilled reading requires us to rapidly retrieve a word's meaning from memory based on a brief exposure to a sequence of arbitrary, abstract visual symbols. Within the lexicon, the mapping from form to meaning takes place at the level of morphology, and rapid semantic retrieval is facilitated by lexical representations that are precise, specific, redundant and flexible. Given that both lexical quality and sensitivity to morphological structure are key components of reading success, can we identify patterns of brain activity that reflect individual differences in these characteristics? To answer this question, we collected event-related potential and response time data from participants as they completed a lexical decision task featuring complex words that varied in the frequency of their morphological components. These data allowed us to determine individual differences in sensitivity to morphological structure. We also quantified individual differences in lexical quality (LQ) via estimates of vocabulary size and spelling performance, both of which are correlated with the precision and stability of lexical representations. We found differences between high LQ and low LQ participants in the N250 ERP component which has been hypothesised to reflect the processing sub-lexical orthographic units. These findings suggest that individual differences in lexical quality and sensitivity to morphological structure are reflected in distinct patterns of brain activity, and opens the avenue for further work to investigate the perceptual and cognitive processes that underlie these differences.
\end{abstract}


%%Each keyword shall be separated by a \sep command. msc classifications shall be provided in the keyword  environment with the commands \MSC. \MSC accepts an optional argument to accommodate future revisions. eg., \MSC[2008]. The default is 2000.

\begin{keyword}
 morphology\sep event-related potentials\sep visual statistical learning \sep individual differences
\end{keyword}

\end{frontmatter}

\linenumbers

\section{Introduction}

Words are the primary means by which language conveys meaning, so learning to decode printed words is an important part of learning to read \citep[e.g.,][]{castles_ending_2018} .  All writing systems consist of abstract visual forms that are mapped onto spoken language units that differ in size—the phoneme for alphabetic systems, the syllable for syllabaries, and the morpheme for logographies. Syllabaries and alphabets gain economy by mapping written units onto the small set of syllables or even smaller set of phonemes  that exist in a language, rather than than the much larger set of morphemes. Alphabets typically use a set of less than one hundred symbols, whereas syllabaries can have several hundred, and logographies can have thousands. Because alphabets map visual symbols to individual speech sounds,  learning the relationship between letters and sounds is necessary and nonnegotiable for beginning readers of any alphabetic writing system.

Alphabetic writing systems differ in terms of orthographic depth, or the consistency of the mapping between letters and sounds. In a strictly alphabetic or ‘shallow’ orthography, one grapheme corresponds to one phoneme, but in a ‘deep’ orthography the mappings between graphemes and phonemes  are many-to-many rather than one-to-one. 

English writing uses a ‘deep’  orthography where the grapheme-to-phoneme mappings are  context-sensitive; they can be affected by factors such as letter position and and the identity of neighbouring graphemes. For example, at the syllabic level, post-vocalic consonants influence vowel pronunciation---the $\langle a\rangle$ in ’hat’ is pronounced differently from the  $\langle a\rangle$ in ‘hate’ as a result of the addition of a syllable final ‘e’---an example of a complex non-adjacent dependency. This leads to a situation in which English spelling is irregular at the level of individual graphemes and phonemes, but much more regular at the level of the spoken and orthographic rimes\footnote{A rime is the part of a syllable which consists of its vowel and any consonant sounds that come after it.} \citep{treiman_special_1995}. Because both graphemes and orthographic rimes play an important role in the pronunciation of printed words, beginning readers must learn how to recognize both types of written units and associate them with their spoken counterparts.

In addition to having a deep orthography, the English writing system also exhibits a trade-off tween phonological explicitness and morphological transparency. The trade-off can be illustrated by the observation that the meanings of groups of letters corresponding to stems are usually preserved in their derivations (e.g., ‘heal’ \textrightarrow ‘health’) and groups of letters corresponding to affixes alter the meanings of stems in consistent ways (e.g., ‘help‑less’, ‘home‑less’, ‘worth‑less’ \textrightarrow ‘‑less’ = ‘without’).  This leads to consistency in the grapheme-to-morpheme mapping at the expense of inconsistency in grapheme-to-phoneme correspondence \citep{ziegler_reading_2005}.

Because of the complexity of English orthography, English-speaking children must acquire a rapid and flexible word recognition system that can accommodate both graphemic to phonemic (G‑P), as well as morpho-orthographic to lexico-semantic (MO‑LS) mappings \citep{ziegler_reading_2005}.  The dual representational character of the  writing system means that both phonological awareness and sensitivity to morphological structure impact reading ability as children progress through school \citep{mahony_reading_2000-1}.  Given the complexity of English orthography it is not possible to convey explicitly all of the relationships between graphemes, phonemes, and morphemes that children need to know in order to become skilled readers; some of this knowledge must be acquired implicitly via the process of statistical learning (SL). 




\section{Front matter}	

The author names and affiliations could be formatted in two ways:
\begin{enumerate}[(1)]
\item Group the authors per affiliation.
\item Use footnotes to indicate the affiliations.
\end{enumerate}
See the front matter of this document for examples. You are recommended to conform your choice to the journal you are submitting to.

\section{Bibliography styles}

There are various bibliography styles available. You can select the style of your choice in the preamble of this document. These styles are Elsevier styles based on standard styles like Harvard and Vancouver. Please use Bib\TeX\ to generate your bibliography and include DOIs whenever available.



% \section*{References}

\bibliography{Individual_Differences}

\end{document}
