\documentclass[]{article}
\usepackage[letterpaper, portrait, margin = 1in]{geometry}
\usepackage[round, sort, semicolon]{natbib} %round brackets, semicolon between citations,
%orders multiple citations into the sequence in which they appear in the list of references
%\usepackage[document]{ragged2e}
%\usepackage{charter}  %changes font to charter
%\usepackage{newcent}  % changes font to new century schoolbook
\usepackage{adforn}	%needed for arrow symbol

%\renewcommand*{\thesubsection}{\arabic{subsection}}
\renewcommand{\labelitemi}{\adfhalfrightarrowhead}
%\renewcommand{\labelitemi}{\S}
\renewcommand{\labelitemii}{\small$\bullet$}
\renewcommand{\labelitemiii}{$\diamond$}
\renewcommand{\labelitemiv}{$\ast$}

\usepackage{hyperref}  	% must be the last command of preamble; not sure why
\hypersetup{colorlinks=false} %set true if you want colored links


%opening


\begin{document}
	\title{Individual Differences Project}
	\author{}
	
	\maketitle
	\tableofcontents
	
	\raggedright
	
	The basic self-teaching hypothesis proposes that a series of successful decoding encounters enables a word to be recognized on the basis of stored information regarding its unique letter pattern.
	
	\section{\citet*{share_phonological_1995}}
	\rule{\textwidth}{.4pt}
	
		\subsection{The development and lexicalization of phonological recoding}	
		\begin{itemize}
			\item Attention to detailed orthographic structure ultimately forms the basis for proficient word recognition
			\item Each successful decoding encounter with an unfamiliar word provides an opportunity to acquire the word-specific orthographic information that is the foundation of skilled word recognition.
			\item Phonological recoding acts as a self-teaching mechanism or built-in teacher enabling a child to independently develop both word-specific and general orthographic knowledge.
			\item A majority of words in natural text will be recognized visually by virtue of their high frequencies, while the smaller number of low-frequency items will provide opportunities for self-teaching with minimal disruption of ongoing comprehension processes.
			\item The process of phonological recoding becomes increasingly ``lexicalized'' in the course of reading development. Simple letter-sound correspondences become modified in the light of lexical constraints imposed by a growing body of orthographic knowledge. The expanding print lexicon alerts the child to regularities beyond the level of simple one-to-one grapheme-phoneme correspondences, such as context-sensitive, positional, and morphemic constraints.
			\item For the beginner, however, an initial set of simple one-to-one correspondences (whose mastery represents no small accomplishment, see Ehri, 1986) may represent the logical point of entry since it offers a minimum number of rules with the maximum generative power.
			\item It should be possible to find reliance on phonological recoding among skilled readers yet direct visual recognition among beginning readers by appropriate manipulation of the item pool.
			\item A judicious mix of visual and phonological recognition processes should, by and large, characterize the word recognition processes of readers at most ability levels, provided of course that reading material is pitched at the appropriate level of difficulty
			\item It is explicit instruction in letter-sound knowledge together with some basic phonemic awareness that bring the decoding possibilities of an alphabetic orthography to a child's attention.
			\item The role of contextual information in resolving decoding ambiguity may also partly explain the surprisingly strong relationship between measures of syntactic awareness (such as sentence correction) and word recognition.
			\item Consonantal correspondences appear the earliest and easiest to acquire. This advantage is generally attributed to their relatively invariant letter-sound relationships in contrast to vowels. Final consonant correspondences may be more difficult. Final consonants may be relatively ``bound'' within the rime unit (vocalic nucleus and final consonant(s)), and hence more difficult to isolate.
			\item Variable correspondences for consonants such as the hard/soft ``c'' alternation present considerable difficulties for young readers. 
			\item The multiple correspondences of vowels require sensitivity to orthographic context, but early decoding skill is based on simple one-to-one correspondences that are relatively insensitive to orthographic and morphemic context.
			\item Later, knowledge of correspondences between orthography and 	phonology becomes increasingly context-sensitive or ``lexicalized''.
			\item The attenuation of regularity effects that accompany increased reading skill can be taken as further support for the ``lexicalization'' hypothesis.
			\item As the orthographic lexicon expands to include a greater number of items and a richer network of connections between these items, the influence of orthographically related items becomes apparent in growing consistency effects
			\item Children's spelling development moves through a sequence of stages characterized by initial adherence to the principle of one letter (or digraph) for each sound prior to acquisition of higher-order regularities such as spelling patterns, positional constraints, and morpheme-based units.
			\item \textbf{An initially incomplete and oversimplified representation of the English spelling-sound system becomes modified and refined in the light of print experience, progressively evolving into a more complete, more accurate and highly sophisticated understanding of the relationships between orthography and phonology}
	\end{itemize}
	
		\subsection{Irregularity and partial decoding}
	
	Because English has multiple ways of representing  every speech sound, every spelling is unique and therefore unpredictable. Thus, both regular and irregular words alike must be dependent on the ability to assimilate word-specific information
	
		\begin{itemize}
			\item The irregularity of printed English is largely restricted to the vowels which may have a marginal role in word recognition.
			\item Most irregular words, when encountered in natural text, have sufficient letter-sound regularity (primarily consonantal) to permit selection of the correct target among a set of candidate pronunciations.
			\item \textbf{Both irregular and regular words appear to depend on the self-teaching afforded by phonological recoding. Furthermore, the role of decoding in learning exception words appears to overshadow the role played by those word-specific visual/orthographic factors generally assumed to be of primary importance precisely because of the presumed inadequacy of letter-sound rules. }
		\end{itemize}

	\section{\citet*{ziegler_reading_2005}}
	\begin{itemize}
		\item The goal of writing is to capture speech for posterity.  The goal of reading is to figure out what was captured.
		\item Each written language represents speech as a series of visual symbols. 
		\item Learning to read is thus fundamentally a process of matching distinctive visual symbols to units of sound.
		\item the relationship between symbol and sound is systematic, whereas the relationship between symbol and meaning is arbitrary.
		\item The process of learning and applying the mapping between symbol and sound has been called phonological recoding.
		\item children need to find shared units in the sorthography and phonology that allow an unambiguous mapping between the two domains.
	\end{itemize}
	
	\section{\citet*{andrews_lexical_2010}}
	
		\subsection{Lexical Precision in Skilled Readers}	
		\begin{itemize}
			\item Perfetti's (1992) lexical quality hypothesis, which argues that a crucial determinant of highly skilled written language processing is the precision of the reader’s lexical representations.
			\item The input features fully determine the representation to be activated and yield fast activation of a single correct representation with minimal activation of competing candidates.
			\item The encapsulation of lexical retrieval processes facilitates reading comprehension because it reduces the need to use context to identify words.
			\item Spelling performance provides the best index of lexical quality because accurate spelling requires precise, word-specific knowledge while many reading tasks can be successfully achieved using only partial information.
			\item Spelling errors are usually phonologically plausible, suggesting they have learned common letter–sound correspondences but failed to acquire the word-specific knowledge required for correct spelling.
			\item \textbf{Good reading comprehension in combination with poor spelling reflects reliance on a reading strategy that uses partial visual cues, in combination with contextual prediction, to recognize words.}
			\item Good readers/poor spellers showed more interference from sentence context in a probe memory task than good readers/ good spellers, particularly when the sentence was presented at a fast rate.
			\item The extent of reliance on lexical processing in good readers/good spellers and good readers/poor spellers should increase across development, as reading experience allows individuals who adopt a more bottom-up reading strategy to refine their lexical representations. 
			\item The masked form priming paradigm is an ideal tool for investigating lexical precision.
			\item  Form priming effects have been reported for neighbor primes that are one letter different from the target word, and transposed letter primes that rearrange the order of the target’s letters
			\item Equally similar primes usually have no impact on responses to nonword targets
			\item This is taken as evidence that similar primes exert their influence by preactivating the lexical representation of the target word. 
			\item Facilitatory masked form priming effects in skilled readers  are restricted to target words that are located in low-density regions of lexical space
			\item The detectors for words in high-density neighborhoods are more narrowly tuned and therefore less vulnerable to being activated by a prime containing a mismatching letter, and tuning is a function of reading experience.
			\item Do similar primes facilitate or inhibit responses to target words?
			\item Nonword neighbor primes give rise to facilitatory effects (Forster et al., 1987).
			\item But other applications of the paradigm produced inhibitory effects of similar primes when the prime was a higher frequency word than the target (Segui \& Grainger, 1990)
			\item But facilitation from both word and nonword primes (e.g., \textit{converse/convenge CONVERGE}) using long, low-N word targets (Forster and Veres, 1998).
			\item Davis and Lupker (2006) found facilitation from nonword neighbor primes but inhibition from word neighbors, particularly when the prime was higher frequency than the target or shared neighbors with the target.
			\item These effects are predicted by  predicted by parallel activation 	models of lexical retrieval that use lateral inhibition between word nodes to achieve lexical selection. 
			\item Word primes produced inhibitory priming relative to a no-prime baseline, whereas nonword primes produced facilitation.
			\item the general IA framework appears to be able to effectively simulate the accumulating evidence about form priming effects in average samples of skilled readers.
	\end{itemize}
	
		\subsection{Experiment 1: Are there differences in the precision of lexical representations among competent readers varying in reading and spelling skill}	
		\begin{itemize}
		\item Priming was compared for high-N and low-N target words; word and nonword primes were also compared
		\item Word neighbor primes were always of higher frequency than their targets and the nonword items were matched with the words on N
		\item Prediction of the lexical quality hypothesis is that precise lexical representations should be less susceptible to priming from similar items and therefore be associated with reduced masked form priming.
		\item If spelling provides the best index of lexical precision, good spellers should be more likely to show inhibitory form priming than poor spellers.
		\item  Reading and spelling were each assessed by two different measures that had previously been administered to between 300 and 600 University of Sydney undergraduates and found to produce relatively normal distributions of scores. A test of vocabulary and a reading span measure of verbal working memory were also included.
		\item Measures were:
		\begin{description}
			\item [Reading Comprehension]  Short passages each  followed by three to five multiple-choice questions assessing comprehension of the passage.
			\item [Reading Speed] participants read a factual text for meaning. To encourage and monitor comprehension of the passage, at intervals of approximately 50 words, participants were required to select from among three words the one that was coherent with the passage.
			\item[Spelling dictation] 20 words were read aloud by the experimenter and included in a sentence to clarify ambiguities. The score for the test was the number of correctly spelled words.
			\item[Spelling recognition] A list of 88 items, half correctly spelled and half misspelled.
			\item[Vocabulary]  30 multiple-choice items selected from the vocabulary subtest of the same instrument as the passage comprehension test, and five alternative options for the meaning of the target.
			\item[Reading span] Developed by Daneman and Carpenter (1980). Comprises 56 unrelated sentences, between 13 and 17 words in length, each ending with a different word. Participants read sentences aloud at their own pace while trying to remember the last word of each sentence in the set.
		\end{description}
		
		\item Stimuli were adapted from those used by Davis and Lupker (2006).
		\item 80 pairs of 4-letter orthographic word neighbors that differed by one letter in any position. The lower frequency member of each pair was designated as the target word, and the higher frequency word served as neighbor prime
		\item Half of the target words were high N (M = 12.5), and the other half were low N (M = 3.6)
		\item Trials:  a forward mask (\#\#\#\#\#\#) for 500 ms, the prime stimulus in lower-case for 50 ms, and the target stimulus in uppercase for 500 ms. The next trial was initiated 500 ms following the response to the previous trial.
		\item High-N words were classified more quickly than low-N words, but the effect was restricted to targets preceded by word primes.
		\item Targets preceded by word primes were classified more slowly than those preceded by nonword primes; 
		\item Nonword primes yielded significant facilitatory priming; this priming effect was not significant by items because it interacted with target nonword neighbor primes facilitated responses to low-N but not high-N targets.
		\item \textbf{\textit{The RT priming effects for the complete sample replicate Forster et al.’s (1987) findings by showing facilitatory neighbor priming only for low-N targets.}} 
		\item \textbf{\textit{However, they also support Davis and Lupker’s (2006) evidence for lexical competition because the facilitatory priming for low-N words only occurred when neighbor primes were nonwords, not words}}
		\item Poorer spelling was associated with lower accuracy for both word targets
		\item Better spelling was associated with a reduced priming effect, and a stronger interaction between priming and N
		\item Poorer spelling was associated with facilitatory priming for both high-N and low-N words
		\item among better spellers, the priming effect for high-N words was inhibitory
		\item \textit{\textbf{Above-average spellers showed inhibitory priming for high-N but not low-N words, regardless of their reading ability, whereas the priming effects for the two groups of below-average spellers were uniformly facilitatory}}
		\item The predictor set accounted for a highly significant 25.6\% of the variance in the RT priming effect for high-N targets but only 2\% of the variance in priming for low-N targets.
		\item Dictation was the strongest unique predictor of inhibitory priming effects for high-N word targets
	\end{itemize}
	
			\subsubsection{Discussion}
			\begin{itemize}
				\item The absence of form priming for high-N words in the total sample was due to averaging over participants who showed opposite patterns of priming for these items. 
				\item Inhibitory RT priming effects for high-N words occurred for better spellers but not poorer spellers, who showed facilitatory priming for both high-N and low-N word targets.
				\item Spelling ability is the most potent unique predictor of inhibitory priming.
				\item Vocabulary did not predict a significant proportion of variance in any of the analyses of RT priming
				\item Vocabulary was, however, a significant unique predictor of average nonword classification time.
				\item This indicates that its lack of contribution to predicting priming was not due to the insensitivity of the vocabulary test.
				\item These results extend and qualify Castles et al.’s (2007) evidence for lexical tuning over the course of reading development.
				\item Precise representations support fast retrieval of prime words and lead to inhibition of their lexical neighbors, including the target word.
				\item Inhibitory effects of word neighbor primes are a straightforward prediction of parallel activation frameworks, like the IA model, which rely on a competitive process to select the lexical representation that best matches the presented stimulus.
				\item \textit{Target preactivation effect}\textendash 	Neighbor primes contain all but one of the target’s letters and therefore preactivate the target word.
				\item \textit{Target neighbor suppression}\textendash They also generate lateral inhibition between competing nodes at the word level.
				\item Nonword neighbor primes produce facilitatory priming due to sublexical overlap without activating any competing word nodes enough to trigger inhibition
				\item Word primes have both facilitatory and inhibitory consequences and therefore yield a net inhibition effect, at least for short words.
				\item The more precise tuning of good spellers’ lexical representations supports rapid activation of the prime word, which, in turn, leads to inhibition of competing neighbors, including the target word.
				\item The impact of lateral inhibition would be enhanced for the short, high-frequency prime/low-frequency target pairs
				\item The effect may be larger for high-N words simply because there are more neighbors to exert competition and counteract the facilitatory impact of letter overlap. 
			\end{itemize}
	
		\subsection{Experiment 2: provide more direct evidence of the differential contributions of target preactivation and target neighbor suppression to individual differences in masked form priming by including partial-word primes}
		\begin{itemize}
			\item \textit{Unambiguous partial-word prime}\textendash activates only the target
			\item \textit{Ambiguous partial-word prime}\textendash activates competing neighbours of the target
			\item In the IA model, \textit{ambiguous partial-word primes} produce strong lateral inhibition of the target. 
			\item \textit{Unambiguous word partial-word primes} produce stronger facilitatory priming of the target.  
			\item  Facilitatory priming from ambiguous primes should also be greater for low-N than high-N words. 
			\item However, priming from unambiguous partial-word primes was relatively insensitive to N as expected if it is due primarily to sublexical overlap.
			\item Five-letter high- and low-N target words were preceded by unambiguous partial-word primes, ambiguous partial-word primes, neighbor word primes, and unrelated word primes
			\item The three primes all share four letters with the target and should therefore yield relatively equivalent target pre-activation
			\item Any differences between them can therefore be attributed to lexical competition
			\item If spelling provides an index of precise lexical representations that support fast inhibition of the prime’s neighbors, both these indices of lexical inhibition should be more marked in better spellers.
			\item Participants were 129 senior undergraduate students 
			\item They found facilitation, relative to unrelated primes, from both ambiguous and unambiguous partial primes
			\item Neighbor primes, on the other hand, produced inhibition relative to unrelated primes but the inhibition only occurred for high-N targets.  Neighbour primes also yielded significant inhibition relative to both unambiguous and ambiguous partial primes
			\item Better spellers exhibited a greater inhibitory effect of neighbor primes relative to both unrelated primes and unambiguous partial primes than poorer spellers
			\item Better spellers showed reduced facilitatory priming for ambiguous relative to unambiguous primes, which was not evident in poorer spellers.
			
		\end{itemize}
	
			\subsubsection{Discussion}
			\begin{itemize}
				\item Experiment 2 replicated the critical novel finding of Experiment 1 that better spelling was associated with stronger inhibitory priming for neighbor primes
				\item partial-word primes behaved like the nonword primes of Experiment 1 and produced facilitatory priming.
				\item The difference between partial primes and neighbor primes was significantly modulated by spelling ability
				\item Rather than showing facilitatory priming only for low-N words, as observed in the overall data for Experiment 1, neighbor priming effects in the overall data for Experiment 2 were restricted to high-N words, and the average priming effect was inhibitory
				\item The overall data for Experiment 2 also contrast with the average word prime conditions of Experiment 1, which showed null effects for both high- and low-N words		
			\end{itemize}	
	
		\subsection{General Discussion}
		\begin{itemize}
			\item There was clear evidence of individual differences in masked form priming within a sample of competent adult readers.
			\item Better spelling was associated with inhibitory priming from masked neighbor primes but only for high-N targets.
			\item Inhibitory neighbor priming was even stronger when neighbor primes were compared with partial-word primes that shared all but one letter with the target.
			\item Neighbor priming is also modulated by individual differences that are best captured by measures of spelling
			\item \textit{\textbf{Masked form priming is sensitive to both facilitatory effects of sublexical overlap and inhibitory effects of lexical competition}}
			\item These two influences appear to be independent because \textbf{spelling ability is selectively associated with lexical competition}
			\item Lexical competition is the major mechanism used to select among co-activated lexical candidates.
			\item Davis and Lupker (2006) evaluated two modifications to the original IA model \textemdash the second modification was to assume that lexical inhibition is stronger between similar rather than dissimilar words
			\item There has been no systematic attempt to extend the IA model to account for individual differences within a language \textemdash  IA simulations assume perfectly precise representations
			\item Lexical inhibition is modulated by formal similarity.
			\item  English four-letter words comprise an unusually dense lexical neighborhood: They have an average of 7.2 neighbors, compared to 2.4 for five-letter words 
			\item the magnitude of the difference between high-N and low-N words was smaller in Experiment 2 than Experiment 1. This weaker N manipulation might account for why the interaction between spelling and neighbor priming in Experiment 2 was not significantly modulated by N.
			\item The different patterns obtained with four- and five-letter words may also indicate that number of neighbors is not the critical variable. 
			\item High- and low-N words differ on many other dimensions of neighborhood structure \textemdash neighbor frequency, number of phonological neighbors, number of shared neighbors, spread of neighbors across letter positions.
			\item Lexical competition is also compatible with parallel distributed processing (PDP) models of visual word recognition
			\item Highly skilled reading is characterized by precise lexical representations that are best indexed by measures of spelling
			\item Precision is assumed to be a property of lexical representations that facilitates activation of the lexical representation corresponding to the sensory input and minimizes activation of competing alternatives
			\item \textbf{How does lexical precision lead to greater lexical competition?}
			\item Perhaps more precise representations are able to be activated more quickly because their orthography is fully specified
			\item Faster accumulation of evidence may allow the representation of the prime word to be activated sufficiently quickly for it to begin to inhibit similar words within the 50-ms prime duration
			\item Poorer spellers showed less priming from neighbor primes than partial primes, suggesting that they were also influenced by lexical competition, but that it may have not developed as quickly as for the better spellers
			\item Another theory\textemdash \textbf{\textit{representational sharpening}}
			\item Neural activity tends to be reduced, rather than enhanced, for repeated stimuli
			\item The neural activity elicited by a repeated stimulus is sharpened by enhancing activation of the elements that discriminate it from other familiar items so that “repetition results in a sparser representation of stimuli”
			\item This process reduces the overall neural activity associated 	with the stimulus. This can potentially lead to detrimental effects of repetition such as repetition blindness but usually enhances behavioral performance for repeated items in priming paradigms because the neural activity that most uniquely defines the item is sustained across repetitions.
			\item The more precise representations of better spellers may have been sharpened through experience to emphasize the features of a word that distinguish it from similar words
			\item The data from both experiments imply that lexical competition is stronger in better spellers, relatively independent of their reading ability
			\item Precise lexical representations are necessary, but not sufficient, for effective comprehension
			\item Poor spellers showed more contextual interference from sentence meaning in a probe memory task than poor readers/good spellers
			\item the latter subgroup effectively retrieve orthographic word forms but are slow or inefficient at retrieving and integrating the semantic information required for sentence comprehension
			\item skilled reading is associated with a more bottom-up reading strategy that involves reduced reliance on context to identify words
			\item Better readers show similar effects of word frequency in predictable and neutral texts, suggesting that they “rely less on surrounding context to bolster their word recognition skills than do less skilled readers		
		\end{itemize}

	
	\section{\citet*{andrews_is_2013-1}}	
	
		\subsection{Background}
		\begin{itemize}
			\item Morphologically complex words provide insight into the organization and retrieval of lexical knowledge.
			\item There are two theoretically important issues at stake
			\begin{itemize}
				\item The utility of morphemes as linguistic building blocks
				\item The psycholinguistic validity of morphemic constituents 
			\end{itemize}
			\item Do morphological priming effects reflect abstract morphological representations or dynamic interactions between orthographic and semantic information during the early stages of lexical retrieval?
			\item A 'transparency effect' is stronger priming for transparent than for opaque pairs. A number of studies have found that the magnitude of the priming effect for opaque pairs is just as large as that for transparent pairs.
			\item This evidence has been taken as support for ‘morpho-graphic decomposition’ accounts of morphological processing an early orthographically-driven segmentation process that parses complex words into affixes and stems.
			\item In an interactive activation model frequently occurring affixes like 're-', '-ed' and'-ing' are represented as sublexical units that can be activated in the early stages of lexical retrieval.
			\item recognition of morphologically complex words begins with a rapid morphemic segmentation based solely on the analysis of orthography
			\item the semantic information responsible for ‘transparency effects’ is attributed to top-down semantic feedback that occurs later in processing
			\item Dual route’ models assume that written words are mapped, in parallel, to morphemic and whole word representations
			\item \citet{taft_sticky_2010} extended the IA framework by adding a layer of ‘lemma’ representations between the form and semantic levels
			\item Lemmas for transparent, but not opaque words, benefit from facilitative connections between related words.
			\item in PDP models,  there are no discrete representational structures corresponding to either words or morphemes.
			\item Words that share both form and meaning become associated with more similar hidden unit representations
			\item Morphological relationships are therefore an emergent consequence of correlations between form and meaning.
			\item many models of morphological processing predict stronger priming for transparent than opaque word pairs.
			\item Empirical demonstrations that they produce equivalent priming have therefore been a focus of considerable scrutiny. 
			\item  a quantitative meta-analysis of 16 studies comparing masked priming for transparent and opaque words showed significantly stronger priming effect for transparent than opaque words, even though the effect was not significant in a number of the individual studies
			\item Definitively demonstrating that there are no semantic effects at very early stages of processing therefore raises the range of statistical issues associated with proving a null hypothesis.
		\end{itemize}
		\subsection{Method}	
			\subsubsection{Possible methodologies}
			\begin{itemize}
				\item Masked priming taps early lexical retrieval processes with little contamination from decision or response strategies
				\item Morphologically complex words facilitate lexical decision responses to their stem morphemes
			\end{itemize}
		
		\subsection{Individual Differences}
		\begin{itemize}
			\item  reliance on average data reflects an implicit ‘uniformity assumption’ that skilled reading is characterized by a uniform, cognitive architecture that can be inferred from average RT data
			\item Further theoretical advances require consideration of individual differences amongst skilled readers, particularly in domains like morphological priming where the empirical results obtained from averaged data have proved inconclusive
			\item  There are systematic individual differences amongst skilled university student readers in both semantic priming \citep{yap_individual_2009} and orthographic priming \citep{andrews_lexical_2010, andrews_not_2012}
			\item Yap showed that the lower vocabulary group showed a larger semantic priming benefit for  difficult, low frequency words, while higher vocabulary group showed a small, consistent benefit from semantic primes for all words.
			\item For low vocabulary participants increased priming reflects greater reliance on prime information for resolving difficult targets.
			\item They attributed the differences to individual differences in the ‘integrity’ or ‘quality’ of their lexical representations
			\item Lexical quality is a construct proposed by Perfetti (1992) to capture variability in the strength and precision of lexical representations
			\item High quality lexical representations that support fast, fluent lexical processing of the prime which confers a small benefit for semantically related targets, even for low frequency words. 
			\item The typical finding in priming studies that average over all participants is that neighbor priming only occurs for Low N words 
			\item However, it has since been found that poor spellers show facilitative neighbor priming while better spellers show inhibitory priming for high N words\textemdash better spelling is selectively associated with lexical competition.
			\item  Masked neighbor priming effects arise from the combined effects of sub-lexical facilitation and lexical competition.
			\item The goal was to evaluate orthographic and semantic contributions to morphological priming by assessing whether spelling and vocabulary exert independent effects on masked priming for opaque and transparently related word pairs.
			\item Spelling indexes the orthographic precision of lexical representations, therefore...
			\item Accurate spelling is associated with a ‘morphographic’ pattern of priming defined by equivalent priming for opaque and transparent pairs
		\end{itemize}

		\begin{itemize}
			\item Participants  assessed on three measures of written language proficiency (a) a test of vocabulary to index semantic knowledge (b) a dictation test and (c) a spelling recognition test.
			\item Transparent, Opaque and Orthographic conditions
			\item RT priming for transparent pairs was significantly stronger than for form controls but not opaque pairs, but opaque pairs did not differ significantly from form pairs either.
			\item Are these average priming effects are modulated by individual differences?
			\item  Two orthogonal measures of individual differences were derived from a principal components analysis
			\begin{itemize}
				\item PC1, indexed by the sum of spelling and vocabulary, tapped general proficiency
				\item PC2,  reflected unique variance differentiating spelling ability from vocabulary with general proficiency partialled out.
			\end{itemize}
			\item 	Higher values on PC2 indicated that spelling scores were relatively higher than vocabulary, while lower values indicate the reverse relationship.
			\item \textit{Are priming effects modulated by individual differences}
			\item  There was no significant main effect of PC1
			\item The association between higher proficiency and faster RT was significantly stronger for transparent target stems,  PC1 did not significantly modulate any of the priming effects.
			\item Superior spelling relative to vocabulary (high PC2) was associated with increased priming for opaque pairs, but reduced priming for transparent pairs. 
			\item Higher vocabulary than spelling (low PC2) was associated with substantially stronger priming for transparent than opaque pairs. 
			\item Superior spelling (high PC2) was associated with greater facilitative priming for opaque pairs, but no change in priming for form pairs.
			\item This contrast was not significant in the models tested on inverse RT, but yielded a robust effect in the models tested on raw RT 
			\item Yap et al. 2009 showed tht individual differences in vocabulary modulate the time course of semantic priming across the RT distribution
			
			\subsection{RT distribution analyses}
			\begin{itemize}
				\item The overall pattern of significantly less priming for form pairs than for the transparent and opaque conditions was evident across every quantile band
				\item The average priming effect decreased across the RT distribution for those of superior spelling but increased for those of superior vocabulary.
				\item In other words for strong spellers, slow items showed less priming than short items while for those with a strong vocabulary, slow items showed more priming.
				\item In the form and transparent conditions superior spelling relative to vocabulary (high PC2) was associated with a reduction in facilitative priming across quantiles, the opposite of the linear increase in facilitative priming for form and transparent pairs shown by  participants whose vocabulary is better than their spelling (low PC2).
				\item priming for opaque pairs steadily increased across the RT distribution for those with superior spelling to vocabulary (high PC2), but decreased for those with better vocabulary than spelling(low PC2) .
				
			\end{itemize}			
			 
		\end{itemize}

	\section{\citet*{yap_individual_2009}}
	
		
		
	\section{\citet*{sanchez-gutierrez_letter_2013-1}}
	\begin{itemize}
		\item The recognition of a printed word such as pot requires the analysis of letter identity as well as of letter order
		\item Letter position coding is  imprecise
		\item The transposed-letter benefit has been interpreted as reflecting a degree of perceptual uncertainty in the coding of letter position
		\item Tolerance for letter transpositions is reduced when the transpositions occur in external, as opposed to internal, positions
		\item Transposed-letter primes facilitate recognition only for the morphologically simple targets, indicating that readers’ tolerance for letter transpositions is reduced if the transpositions arise across a morpheme boundary.
		\item \citet*{dunabeitia_transposed-letter_2007} contrasted masked-priming effects on morphologically complex versus morphologically simple targets
		\item Tansposed-letter primes yielded facilitation on target recognition (relative to substitution primes) only for morphologically simple targets.
		\item Orthographic coding demands sufficient precision to identify morphemic units, perhaps in order to facilitate the morphemic segmentation processes thought to characterize the initial stages of visual word recognition 
		\item In contrast, Rueckl and Rimzhim (2011) reported five masked-priming experiments in which they demonstrated persuasively that facilitation is observed even in cases in which the transposed letters straddle a morpheme boundary and that this facilitation is equivalent when the transposed letters arise within a stem or across a morpheme boundary
		\item Letter transpositions can have very different effects on the recognition of words in languages with qualitatively different morphological characteristics
		\item Spanish is characterized by far greater morphological diversity and productivity than English
		\item precision in the orthographic coding of the morphemic boundary may be more important in Spanish than it is in English
		\item Experiment 1a (Spanish)  responses to words preceded by TL primes were faster than those to words preceded by RL primes; his main effect was not modulated by position
		\item Experiment 1b (English) responses to words preceded by TL primes were faster than those to words preceded by RL primes ; this main effect was not modulated by position
		\item The results were unambiguous in showing transposed-letter benefits in both Spanish and English that were not modulated by the position of the transposed letter in the prime stimulus.
	\end{itemize}
	
	\section{\citet*{dunabeitia_revisiting_2014}}
	\begin{itemize}
		\item \citet*{sanchez-gutierrez_letter_2013} found masked TL priming effects of similar magnitude for betweenand within-morpheme transpositions
		\item The diverging results for TL manipulations across morphemic boundaries could potentially stem from individual differences in reading skills on morphological decomposition and morpho-orthographic interactions.
		\item Although good spellers with average vocabulary (i.e., readers with an orthographic profile) showed significant priming effects for opaque pairs, high-vocabulary participants with average spelling skills (i.e., readers with a semantic profile) showed minimal priming for opaque relationships.
		\item For faster readers, TL priming effects were greater for within-morpheme than for between-morpheme transpositions, whereas no such difference occurred for the slower readers.
		\item Reexamined the data using a centilebased analysis of the RT distribution and found that the TL effects for within-morpheme manipulations were \textbf{\textit{unaffected}} by the speed of response, while the TL effects for between-morpheme manipulations \textbf{\textit{increased}} as a function of the speed of response
	\end{itemize}
		\subsection{Discussion}
		\begin{itemize}
			\item It is possible to obtain (fairly small) masked-priming effects when the transposed-letter manipulation crosses the affix boundary (around 10 ms, when averaging faster and slower readers)
			\item When the individual differences of the participants in their reading speeds were taken into account, clear-cut differences emerged in the magnitudes of the TL priming effects for betweenand within-morpheme manipulations.
			\item The group of faster participants only showed TL priming effects in within-morpheme conditions, but not between morphemes
			\item Faster readers (who are, potentially, the best readers) are highly sensitive to morpho-orthographic interactions: the magnitude of the between-morpheme TL effect was half the size of that of the within-morpheme TL effect
			\item  TL effects across morphemic boundaries can be captured better by a continuum that may be linked to individual differences in reading performance and/or to participants’ reading styles/ profiles
			\item Faster readers follow a reading style based on fast-acting automatic morphological decomposition processes
			\item  Slower readers are hypothesized to focus less on orthographic information, probably basing their reading profile on semantic-based pieces of information
			\item Evidence for this is that the concreteness effect is inversely correlated with the orthographic skills of the readers
		\end{itemize}
		
	
	\bibliographystyle{plainnat}
	\bibliography{ind_diff_notes}	
	
\end{document}
