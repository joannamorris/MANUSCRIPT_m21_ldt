\documentclass[review]{elsarticle}

\usepackage{lineno,hyperref}
\modulolinenumbers[5]
%\usepackage{natbib}

\journal{Journal of \LaTeX\ Templates}

%%%%%%%%%%%%%%%%%%%%%%%
%% Elsevier bibliography styles
%%%%%%%%%%%%%%%%%%%%%%%
%% To change the style, put a % in front of the second line of the current style and
%% remove the % from the second line of the style you would like to use.
%%%%%%%%%%%%%%%%%%%%%%%

%% Numbered
%\bibliographystyle{model1-num-names}

%% Numbered without titles
%\bibliographystyle{model1a-num-names}

%% Harvard
%\bibliographystyle{model2-names.bst}\biboptions{authoryear}

%% Vancouver numbered
%\usepackage{numcompress}\bibliographystyle{model3-num-names}

%% Vancouver name/year
%\usepackage{numcompress}\bibliographystyle{model4-names}\biboptions{authoryear}

%% APA style
 \bibliographystyle{model5-names}\biboptions{authoryear}

%% AMA style
%\usepackage{numcompress}\bibliographystyle{model6-num-names}

%% `Elsevier LaTeX' style
% \bibliographystyle{elsarticle-num}
%%%%%%%%%%%%%%%%%%%%%%%

\begin{document}

\begin{frontmatter}

\title{Elsevier \LaTeX\ template\tnoteref{mytitlenote}}
\tnotetext[mytitlenote]{Fully documented templates are available in the elsarticle package on \href{http://www.ctan.org/tex-archive/macros/latex/contrib/elsarticle}{CTAN}.}

%%% Group authors per affiliation:
%\author{Elsevier\fnref{myfootnote}}
%\address{Radarweg 29, Amsterdam}
%\fntext[myfootnote]{Since 1880.}

%% or include affiliations in footnotes:
\author[mymainaddress]{Joanna Morris\corref{mycorrespondingauthor}}
\cortext[mycorrespondingauthor]{Corresponding author}
\ead{jmorris6@providence.edu}

\author[mymainaddress]{Emma Kealey}
\ead{ekealey@providence.edu}

\address[mymainaddress]{Department of Psychology, Providence College, Providence, RI, USA}


\begin{abstract}
Skilled reading requires us to rapidly retrieve a word's meaning from memory based on a brief exposure to a sequence of arbitrary, abstract visual symbols. Within the lexicon, the mapping from form to meaning takes place at the level of morphology, and rapid semantic retrieval is facilitated by lexical representations that are precise, specific, redundant and flexible. Given that both lexical quality and sensitivity to morphological structure are key components of reading success, can we identify patterns of brain activity that reflect individual differences in these characteristics? To answer this question, we collected event-related potential and response time data from participants as they completed a lexical decision task featuring complex words that varied in the frequency of their morphological components. These data allowed us to determine individual differences in sensitivity to morphological structure. We also quantified individual differences in lexical quality (LQ) via estimates of vocabulary size and spelling performance, both of which are correlated with the precision and stability of lexical representations. We found differences between high LQ and low LQ participants in the N250 ERP component which has been hypothesised to reflect the processing sub-lexical orthographic units. These findings suggest that individual differences in lexical quality and sensitivity to morphological structure are reflected in distinct patterns of brain activity, and opens the avenue for further work to investigate the perceptual and cognitive processes that underlie these differences.
\end{abstract}


%%Each keyword shall be separated by a \sep command. msc classifications shall be provided in the keyword  environment with the commands \MSC. \MSC accepts an optional argument to accommodate future revisions. eg., \MSC[2008]. The default is 2000.

\begin{keyword}
 morphology\sep event-related potentials\sep visual statistical learning \sep individual differences
\end{keyword}

\end{frontmatter}

\linenumbers

\section{Introduction}

Words are the primary means by which language conveys meaning, hence learning to decode printed words is an important part of learning to read \citep[e.g.,][]{castlesEndingReadingWars2018}.  Although all writing systems consist of abstract visual symbols that are mapped onto spoken language units, orthographies differ with respect to \textit{which} spoken unit is mapped to a visual symbol; alphabetic systems map symbols to phonemes, syllabaries  map them to syllables, and logographies to words or morphemes \footnote{Chinese is typically cited as the canonical example of a logographic system, but although it may have originated as a logography, in its current incarnation it is more accurately classified as a morpho-syllabic system \citep{raynerHowPsychologicalScience2001}}.  The choice of spoken linguistic unit has consequences for how reading proceeds. 

Because alphabets map visual symbols to individual speech sounds rather than to units of meaning as logographies do,  learning the relationship between letters and sounds is necessary and nonnegotiable for beginning readers of any alphabetic writing system.  However, learning this mapping can be difficult because systems differ in terms of orthographic depth, or the consistency of the mapping between letters and sounds. In a strictly alphabetic or ‘shallow’ orthography, one grapheme corresponds to one phoneme, but in a ‘deep’ orthography the mappings between graphemes and phonemes  are many-to-many rather than one-to-one. 

English writing uses a ‘deep’  orthography where the grapheme-to-phoneme mappings are  context-sensitive, and  can be affected by factors such as letter position and and the identity of neighbouring graphemes. For example, at the syllabic level, post-vocalic consonants influence vowel pronunciation---the $\langle a\rangle$ in `hat’ is pronounced differently from the  $\langle a\rangle$ in ‘hate’ as a result of the addition of a syllable final ‘e’---an example of a complex non-adjacent dependency. This leads to a situation in which English spelling is irregular at the level of individual graphemes and phonemes, but much more regular at the level of the spoken and orthographic rimes\footnote{A rime is the part of a syllable which consists of its vowel and any consonant sounds that come after it.} \citep{treimanSpecialRoleRimes1995}. Because both graphemes and orthographic rimes play an important role in the pronunciation of printed words, beginning readers must learn how to recognize both types of written units and associate them with their spoken counterparts.

In addition to having a deep orthography, the English writing system also exhibits a trade-off between phonological explicitness and morphological transparency. The trade-off can be illustrated by the observation that the meanings of groups of letters corresponding to stems are usually preserved in their derivations (e.g., ‘heal’ \textrightarrow ‘health’) and groups of letters corresponding to affixes alter the meanings of stems in consistent ways (e.g., ‘help‑less’, ‘home‑less’, ‘worth‑less’ \textrightarrow ‘‑less’ = ‘without’).  This leads to consistency in the grapheme-to-morpheme mapping at the expense of inconsistency in grapheme-to-phoneme correspondence \citep{zieglerReadingAcquisitionDevelopmental2005,rastlePlaceMorphologyLearning2019}.

Because of the complexity of English orthography, English-speaking children must acquire a rapid and flexible word recognition system that can accommodate graphemic to phonemic (G‑P) mappings from symbols to phonemes, graphemic to syllabic (G-S) mappings from symbols to syllables and sub-syllabic units such as onsets and rimes, as well as  morpho-orthographic to lexico-semantic (MO‑LS) mappings from symbols to units of meaning such as morphemes  and words.  The multi-representational character of the  writing system means that  phonological awareness, syllabic awareness and sensitivity to morphological structure impact reading ability as children progress through school \citep{mahonyReadingAbilitySensitivity2000}.  

Children are rarely explicitly taught all of the relationships between graphemes, phonemes, syllables and morphemes that they must know in order to become skilled readers; some of this knowledge must be acquired implicitly via the process of  statistical learning (SL), the mechanism by which we extract repeated patterns from our environment. SL enables the detection of probabilistic correspondences between letters and phonemes as well as the detection of regularities that exist among letters. For example, an orthographic unit, such as a word (e.g. `bird') or affix (e.g. `-er'), is defined by higher transitional probabilities between letters within a unit than between letters spanning two distinct units.	 Sensitivity to probabilistic orthography-to-phonology mappings follows a developmental trajectory throughout the elementary and middle school years \citep{treimanSpellingStatisticalLearning2006} and  it is  a significant predictor of reading ability, but there is substantial variation in SL performance among both children and adults \citep{arciuliStatisticalLearningRelated2012}.   Do these differences in SL contribute to individual differences in  reading skill?

\subsection{Quality of lexical representations}




Efficient skilled reading depends on the rapid retrieval of a  word’s phonological and semantic information from memory based on a brief exposure to its orthographic representation. \citep{rastlePlaceMorphologyLearning2019, castlesHowDoesOrthographic2006}. Rapid and reliable retrieval is  facilitated by high-quality semantic, phonological and orthographic \citep{perfettiLexicalQualityRevisited2017} representations, where 'high-quality' is defined as representations that are precise, specific, redundant and flexible \citep{perfettiReadingAbilityLexical2007}. 

\citet{yapIndividualDifferencesJoint2009} propose that vocabulary knowledge (i.e., knowledge of word forms and word meanings) is a reasonable proxy for lexical quality as there is evidence that vocabulary size is correlated with the precision and stability of lexical representations \citep{perfettiReadingAbilityLexical2007, perfettiLexicalQualityHypothesis2002, kiddIndividualDifferencesStatistical2016,kinoshitaAdditiveInteractiveEffects2006, kinoshitaHowLexicalDecision2006}  Along with vocabulary size, spelling performance provides a good index of lexical quality because while many reading tasks can be successfully achieved using only partial information, accurate spelling requires precise, word-specific knowledge (Perfetti, 1992).


\section{Front matter}	

The author names and affiliations could be formatted in two ways:
\begin{enumerate}[(1)]
\item Group the authors per affiliation.
\item Use footnotes to indicate the affiliations.
\end{enumerate}
See the front matter of this document for examples. You are recommended to conform your choice to the journal you are submitting to.

\section{Bibliography styles}

There are various bibliography styles available. You can select the style of your choice in the preamble of this document. These styles are Elsevier styles based on standard styles like Harvard and Vancouver. Please use Bib\TeX\ to generate your bibliography and include DOIs whenever available.



% \section*{References}

\bibliography{Individual_Differences}

\end{document}
